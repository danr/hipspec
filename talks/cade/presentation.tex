%!TEX TS-program = xelatex
%!TEX encoding = UTF-8 Unicode

% alias hipspec='hipspec --auto --cg --verbosity=40 --smt-data-types'

\documentclass[serif,professionalfont]{beamer}

\usepackage{tikz}
\usetikzlibrary{shapes,arrows,calc}
\usepackage{verbatim}

% \usepackage[utf8]{inputenc}
% \usepackage[T1]{fontenc}
%\usepackage{fixltx2e}
%\usepackage{graphicx}
%\usepackage{longtable}
%\usepackage{float}
%\usepackage{wrapfig}
%\usepackage{soul}
%
%\usepackage{textcomp}
%\usepackage{marvosym}
%\usepackage{wasysym}
%\usepackage{latexsym}
\usepackage{amssymb}

\usepackage{hyperref}

\usepackage{mathpartir}
\usepackage{color}
%\tolerance=1000
\usepackage{inconsolata}
\usepackage{amsmath}
\usepackage{array}
\providecommand{\alert}[1]{\textbf{#1}}

\definecolor{Purpleee}{RGB}{140,20,140}
\definecolor{PurpleL}{RGB}{255,170,255}
\definecolor{MyGreen}{RGB}{5,95,5}
\setbeamercolor{title}{fg=Purpleee}
\setbeamercolor{frametitle}{fg=Purpleee}
\setbeamercolor{structure}{fg=Purpleee}


\TeXXeTstate=1
\usepackage{mathspec,xltxtra,xunicode}
% \setsansfont{Gill Sans}

% \setmainfont[Scale=1]{Gill Sans}
%\setmonofont[Scale=0.8]{Monaco}
\setmonofont{Inconsolata}

% \setmathsfont(Digits,Latin,Greek){Gill Sans}

\usefonttheme[onlymath]{serif}

%\usepackage{fontspec}
%\defaultfontfeatures{Mapping=tex-text}
%\setsansfont[Ligatures={Common}]{Futura}


\usepackage{listings}

\lstnewenvironment{codex}[1][]%
  {\noindent
   \minipage{\linewidth}
   \vspace{0.2\baselineskip}
%   \vspace{-0.4\baselineskip}
   \lstset{basicstyle=\ttfamily,
%           frame=single,
           language=Haskell,
           keywordstyle=\color{black},
           #1}}
  {%\vspace{-0.8\baselineskip}
   \endminipage}

\title{{\Huge HipSpec}}
\subtitle{{\LARGE Automating Inductive Proofs \\ using Theory Exploration}}
\institute{Chalmers University of Technology}

\author{Dan Ros\'en
     \vspace{\baselineskip} \\
     Koen Claessen, Moa Johansson, Nicholas Smallbone}
 \date{} % June 4, 2013}

\newcommand\fa[1]{ \forall \, #1 . \,}
\newcommand\faa[2]{ \forall \, #1 , #2 . \,}
\newcommand\faaa[3]{ \forall \, #1 , #2 , #3 . \,}
\newcommand\faaaa[4]{ \forall \, #1 , #2 , #3 , #4 . \,}
\newcommand\up[0]{\vspace{-\baselineskip}}
\newcommand\dn[0]{\vspace{\baselineskip}}
\newcommand\hs[1]{\texttt{#1}}

\newcommand\x[0]{\hs{x}}
\newcommand\xs[0]{\hs{xs}}
\newcommand\ys[0]{\hs{ys}}
\newcommand\xxs[0]{\hs{x:xs}}
\newcommand\nil[0]{\hs{[]}}
\newcommand\p[1]{\textsf{P}(#1)}

\newcommand{\highlight}[1]{\colorbox{PurpleL}{\ensuremath{#1}}}

\begin{document}

\maketitle

\defverbatim{\rotatedef}{%
\begin{verbatim}
    rotate :: Nat -> [a] -> [a]
    rotate Zero     xs     = xs
    rotate (Succ n) []     = []
    rotate (Succ n) (x:xs) = rotate n (xs ++ [x])
\end{verbatim}
}

\defverbatim{\rotateex}{%
\begin{verbatim}
    rotate 1 [1,2,3,4] = [2,3,4,1]
    rotate 2 [1,2,3,4] = [3,4,1,2]
    rotate 3 [1,2,3,4] = [4,1,2,3]
    rotate 4 [1,2,3,4] = [1,2,3,4]
\end{verbatim}
}

\defverbatim{\rotateprop}{%
\begin{equation*}
    \fa{\xs} \hs{rotate (length xs) xs} = \hs{xs}
\end{equation*}
}

\defverbatim{\rotatestep}{%
\begin{align*}
    \hs{rotate (length (a:as)) (a:as) } & = \\
    \hs{rotate (S (length as)) (a:as) } & = \\
    \hs{rotate (length as) (as ++ [a])} & =
\end{align*}
}

\defverbatim{\rotateih}{%
\begin{equation*}
    \hs{rotate (length as) as} = \hs{as}
\end{equation*}
}

\defverbatim{\rotategen}{%
\begin{equation*}
    \faa{\xs}{\ys} \hs{rotate (length xs) (xs ++ ys)} = \hs{ys ++ xs}
\end{equation*}
}

\defverbatim{\rotategenstep}{%
 \begin{align*}
     & \hs{rotate (length (a:as)) (a:as ++ bs)} & = & \\
     & \hs{rotate (S (length as)) (a:as ++ bs)} & = & \\
     & \hs{rotate (length as) (as ++ bs ++ [a]) } & = & \{ \mathsf{IH} \} \\
     & \hs{bs ++ [a] ++ as                      } & = & \\
     & \hs{bs ++ (a:as)                         } & &
\end{align*}
}

\defverbatim{\rotategenih}{%
\begin{equation*}
    \fa{\ys} \hs{rotate (length as) (as ++ ys)} = \hs{ys ++ as}
\end{equation*}
}

\defverbatim[colored]{\stuck}{%
\begin{center}
  {\color{red} Stuck!}
\end{center}
}

\defverbatim{\sortedprop}{%
\begin{equation*}
  \fa{\xs}  \hs{sorted (isort xs)} = \hs{True}
\end{equation*}
}

\defverbatim{\insertlemma}{%
\begin{equation*}
  \fa{\xs}  \hs{sorted xs} = \hs{True} \Rightarrow
            \hs{sorted (insert x xs)} = \hs{True}
\end{equation*}
}


%\begin{frame}
%    \frametitle{Setting}
%    \begin{itemize}
%        \item Automating proofs by induction of functional programs
%        \item Terminating subset of Haskell
%    \end{itemize}
%\end{frame}

\begin{frame}[fragile]
    \frametitle{Rotate Example}

    \begin{overprint}
    \rotatedef
    \action<2->{\dn\rotateprop}

    \onslide<1> \rotateex

    \onslide<2> \rotateex

    \onslide<3>
        hypothesis: \rotateih
        conclusion: \rotatestep

    \onslide<4>
        hypothesis: \rotateih
        conclusion: \rotatestep
        \stuck

    \end{overprint}

\end{frame}

\begin{frame}
    \frametitle{Rotate-length Helper Lemma}

    \rotategen

    \pause

    conclusion:

    \rotategenstep

    hypothesis:

    \rotategenih

    \pause

    Bundy, Basin, Hutter, Ireland: automated induction challenge in
    \emph{Rippling: meta-level guidance for mathematical reasoning}

\end{frame}

% Rotate demo

% We're doing theory exploration right now, to conjecture useful lemmas,
% done by a tool called QuickSpec,
% Now we've started proving things by induction, and these properties
% were not specified in the source code but were conjectured during TE
% oh there's our property being proved! hooray!

\begin{comment} % old setting
  Prove properties of functional programs using rewriting and
  induction.

  {\color{Purpleee} Problem:} Som properties require lemmas to be proved, that

  \begin{itemize}
    \item Needs to be conjectured,
    \item Requires induction to be proved, and
    \item Might require lemmas themselves
  \end{itemize}
\end{comment}


\begin{frame}
    \frametitle{QuickSpec: the Theory Exploration Phase}
    Generates well-typed terms up to some depth:
    \dn
    \newcommand\vb[1]{{\tt #1}}

\begin{tabular}{>{\footnotesize}l >{\footnotesize}l >{\footnotesize}l >{\footnotesize}l}
\vb{[]}&
\vb{[]++[]}&
\vb{qrev [] []}&
\vb{qrev (rev xs) []}\\
\vb{qrev [] (rev xs)}&
\vb{qrev (rev xs) ys}&
\vb{qrev [] xs}&
\vb{qrev xs []}\\
\vb{[]++qrev xs ys}&
\vb{qrev [] (xs++ys)}&
\vb{(x:xs)++[]}&
\vb{qrev xs ys++[]}\\
\vb{qrev (x:[]) xs}&
\vb{qrev [] (x:xs)}&
\vb{rev []}&
\vb{rev (qrev ys xs)}\\
\vb{rev (rev xs)}&
\vb{[]++rev xs}&
\vb{rev xs}&
\vb{rev xs++ys}\\
\vb{xs}&
\vb{[]++xs}&
\vb{xs++[]}&
\vb{(xs++ys)++[]}\\
\vb{[]++(xs++ys)}&
\vb{xs++ys}&
\vb{(x:[])++xs}&
\vb{xs++(x:[])}
\end{tabular}

\end{frame}

\begin{frame}[fragile]
    \frametitle{Partitioning into Equivalence Classes}

    \tikzstyle{eqblock} = [rectangle, draw=Purpleee, thick, text width=4.75em]
\newcommand\vb[1]{\mbox{{\tt #1}}}
\makebox[\textwidth][c]{
\begin{tikzpicture}

\node at (0,2) [eqblock,text width=90] (src) {
\vb{xs}\\
\vb{xs++[]}\\
\vb{[]++xs}\\
\vb{qrev [] xs}\\
\vb{rev (rev xs)}\\
\vb{qrev (rev xs) []}\\
};


\node at (0,-2) [eqblock,text width=90] {
\vb{xs++ys}\\
\vb{qrev (rev xs) ys}\\
\vb{[]++(xs++ys)}\\
\vb{qrev [] (xs++ys)}\\
\vb{(xs++ys)++[]}\\
};

\node at (4,2) [eqblock,text width=60] {
\vb{[]}\\
\vb{rev []}\\
\vb{qrev [] []}\\
\vb{[]++[]}\\
};

\node at (4,-2)  [eqblock,text width=80] {
\vb{x:xs}\\
\vb{[]++(x:xs)}\\
\vb{qrev [] (x:xs)}\\
\vb{(x:xs)++[]}\\
\vb{(x:[])++xs}\\
\vb{qrev (x:[]) xs}\\
};

\node at (8,2)  [eqblock,text width=120] {
\vb{qrev xs ys}\\
\vb{rev (qrev ys xs)}\\
\vb{rev xs++ys}\\
\vb{[]++qrev xs ys}\\
\vb{qrev [] (qrev xs ys)}\\
\vb{qrev xs ys++[]}\\
\vb{qrev (qrev ys xs) []}\\

};

\node at (8,-2)  [eqblock,text width=100] {
\vb{rev xs}\\
\vb{qrev xs []}\\
\vb{[]++rev xs}\\
\vb{qrev [] (rev xs)}\\
};

\end{tikzpicture}
}


    % read off the equations from the classes

    % completeness up to some depth

    % false properties are not a problem given that
    % the function definitions do not have tricky corner cases
    % that the random value generators do not cover
\end{frame}

\begin{frame}[fragile]
  \frametitle{Hip: The Haskell Inductive Prover}

  \begin{itemize}
    \item Translate to typed first order logic
    \item Apply structural induction
          % in many different ways

  \end{itemize}

  Also supports higher-order functions and partial application
\end{frame}

\begin{frame}
  \frametitle{Overview of HipSpec}
  \newcommand{\h}[2]{\action<#1->{#2}}

% Define block styles
\tikzstyle{bigblock}   = [rectangle, draw=Purpleee, thick,
                          text width=4.75em, text centered,
                          minimum height=10em,
                          rounded corners=.8ex]

\tikzstyle{smallblock} = [rectangle,
                          text width=6em, text centered,
                          minimum height=3em,thick,
                          draw=Purpleee,rounded corners=.8ex]

\tikzstyle{arr} = [->,>=stealth',semithick]

\makebox[\textwidth][c]{
\begin{tikzpicture}

    % Place nodes
    \node at (-0.75,0)     [bigblock] (src)  {Haskell Source};
    \pause

    \node at (4,1.5)  [smallblock] (conj) {Conjectures};
    \draw [arr] (src)  to [bend left=15] node [above]
       {
         QuickSpec
       } (conj);

    \pause
    \node at (4,-1.5) [smallblock] (thy)  {First-Order Theory};
    \draw [arr] (src)  to [bend left=-15] node [below,text centered,text width=5em]
       {
         Translation (Hip)
       } (thy);
    \pause
    \node at (8.5,0)     [smallblock,text width=8em] (atp)  {Theorem Prover \\ {\small (E, Vampire, Z3...)}};

    \node at (4.5,0) (text) {Induction (Hip)};


    \draw [arr] (conj) to [bend right=17]  (atp);
    \draw [arr] (thy)  to [bend right=-17] (atp);

    \pause
    \draw [arr] (atp)  to [bend left=15]
        node [right,near start] {\texttt{ Theorem}}
        node [below right,very near end] {Extend theory} (thy);

    \pause
    \draw [arr] (atp)  to [bend left=-15]
        node [right,near start] {\texttt{ Timeout}}
        node [above right,very near end] {Open conjecture} (conj);

\end{tikzpicture}
}



\end{frame}


\begin{frame}[fragile]
  \frametitle{Prioritising Equations}

  \begin{overprint}
  \begin{minipage}{.5\linewidth}
  \begin{enumerate}
      \item Call graph
      \action<3->{\item Size of term}
      \action<4->{\item Number of variables}
  \end{enumerate}
  \end{minipage}%
  \begin{minipage}{.5\linewidth}
      \action<1-2>{\tikzstyle{cgblob} = [rectangle, rounded corners=10, draw=Purpleee, thick, text width=40, text height=10, text centered]
\tikzstyle{cgarr} = [->,>=stealth',semithick]
%\makebox[\textwidth][c]{\
\begin{tikzpicture}

    \node at (0,1.6) [cgblob] (app) {\hs{(++)}\vphantom{g}} ;

    \node at (2.4,1.6) [cgblob] (len) {\hs{length}\vphantom{g}} ;

    \node at (1.2,0)  [cgblob] (rot) {\hs{rotate}\vphantom{g}} ;

    \draw [cgarr] (rot) to (app) {} ;
    \draw [cgarr] (rot) to (len) {} ;

\end{tikzpicture}
%}
}
  \end{minipage}

  \dn
  \dn

  \onslide<2>


  \begin{verbatim}
  xs++[] = xs
  length (xs++ys) = length (ys++xs)
  rotate (length xs) (xs ++ ys) = ys ++ xs
  \end{verbatim}

  \onslide<3>
  \begin{verbatim}
  xs++[] = xs
  (xs++ys)++zs = xs++(ys++zs)
  \end{verbatim}

  \onslide<4>
  \begin{verbatim}
  (xs++ys)++zs = xs++(ys++zs)
  (xs++xs)++ys = xs++(xs++ys)
  (xs++xs)++xs = xs++(xs++xs)
  \end{verbatim}
  \end{overprint}

\end{frame}

% so, how well does this work, well let's look at our..

\begin{frame}
  \frametitle{Evaluation Results I}

1st test suite from \emph{Case-analysis for Rippling and Inductive Proof}:

\begin{center}
\begin{tabular}{c>{\hskip0.5em}c>{\hskip0.5em}c>{\hskip0.5em}c>{\hskip0.5em}c>{\hskip0.5em}c}
   \#Props  & HipSpec & Zeno & ACL2s & IsaPlanner & Dafny \\
\hline
    85      & 80      & 82   & 74    & 47         & 45    \\
\end{tabular}
\end{center}

\begin{itemize}
  \item Quite easy: around 60 provable without lemmas
  \item Some require conditional lemmas \\ (can't generate with QuickSpec)
\end{itemize}

\end{frame}

\begin{frame}
  \frametitle{Evaluation Results II}

2nd test suite from \emph{Productive Use of Failure in Inductive Proof}:

\begin{center}
\begin{tabular}{c>{\hskip0.5em}c>{\hskip0.5em}c>{\hskip0.5em}c}
    \#Props  & HipSpec & CLAM & Zeno \\
\hline
    50       & 44      & 41   & 21
\end{tabular}
\end{center}

\begin{itemize}
  \item Harder test suite: need lemmas and generalisations
\end{itemize}

\end{frame}

% so what is it that we cannot prove? let's look at this

\begin{frame}[fragile]
  \frametitle{Conjecturing Conditionals}

  \sortedprop

  \begin{verbatim}
    isort :: [Nat] -> [Nat]

    insert :: Nat -> [Nat] -> [Nat]

    sorted :: [Nat] -> Bool
  \end{verbatim}

  \pause

  % It's one property that we currently cannot prove fully automatically

  Requires:

  \insertlemma

  % How do we conjecture such properties?
  % Currently, QuickSpec is implemented at looking at equivalence classes that
  % are unconditionally true... would like to have some kind of precondition on
  % certain equivalence classes to talk about when xs is always sorted, for
  % instance. Another problem might be that it is so uncommon to generate
  % sorted lists, so we will probably need to incorporate other ongoing work on
  % generating random values satisfying some predicate.

\end{frame}

\defverbatim{\tricky}{%
  \begin{equation*}
      \faa{\hs{i}}{\xs} \hs{rev (drop i xs)} = \hs{take (length xs - i) (rev xs)}
  \end{equation*}
}
\defverbatim{\trickylemmas}{%
  Required lemmas:

  \begin{center}
  \begin{verbatim}
  length (drop x xs)          = length xs - x
  length (rev xs)             = length xs
  take x xs ++ drop x xs      = xs
  rev xs ++ rev ys            = rev (ys++xs)
  take (length xs) (xs ++ ys) = xs
  \end{verbatim}
  \end{center}
}

% so these theorem provers work top-down

\begin{frame}
  \frametitle{Two Approaches to Lemma Discovery}

    \begin{overprint}

    \begin{itemize}
      \item Top-down: Rippling/critics-based provers, ACL2, Zeno
      \onslide<5->{\item Bottom-up: IsaCoSy, IsaScheme, HipSpec}
    \end{itemize}

    % looking at the stuck proof (maybe change to another example)

    \onslide<2>
    \rotateprop \rotatestep \stuck

    % this is one example where I think it would be really hard

    \onslide<3> \tricky

    % so why hard... well look at the lemmas you need

    \onslide<4->
    \tricky
    \dn
    \trickylemmas
    \end{overprint}

\end{frame}

% "So the tools IsaCoSy and IsaScheme can conjecture a theory,
% but not fully automatically when proving a user-stated property,
% which is what HipSpec can. But HipSpec can also be run without
% specifying what to prove! Let's have a look at the output
% of when rotate was run"

% See, the property we wanted to prove was conjectured by quickspec.
% And of course it is! It is within the depth limit, so this is fully
% expected. So we could run this example without specifying that this
% is the property we want to show, and it will still prove it!
% But instead of looking at that, let's look at another common
% problem in this literature, namely rev/qrev...
% (explanation + BAM)
% Another example is this, so we have unary/peano nats and plus
% and multiplication defined over this. So we should expect
% that hipspec can prove that this is a commutative semiring...
% let's try...

\begin{frame}


\frametitle{Theory Exploration Results}
\begin{itemize}
\item Produce background theory comparable to human.
\item Comparison with Isabelle libraries.
\item \textbf{HipSpec runs in minutes.} IsaCoSy/IsaScheme sometimes hours.
\end{itemize}
 \begin{table}[htd]
\begin{center}
\begin{tabular}{l|c|c|c||c}
%\hline
                           & HipSpec & IsaCoSy & IsaScheme & Isabelle\\
 \hline
 \textbf{\#Thms Naturals}  & 10      & 16      & 16*       & 12\\
 \hline
  Precision                & 80\%    & 63\%    & 100\%*    & -\\
  Recall                   & 73\%    & 83\%    & 46\%*     & -\\
\hline
 \textbf{\#Thms Lists}     & 10      & 24      & 13        & 9 \\
  \hline
  Precision                & 90\%    & 38\%    & 70\%      & -\\
  Recall                   & 100\%   & 100\%   & 100\%     & -\\
%\hline
\end{tabular}
\end{center}
\caption{Theory Exploration results. IsaScheme was evaluated on a natural number theory also including exponentiation.}
\label{tab:expl}
\end{table}%

\end{frame}

% \begin{frame}
%   \frametitle{}
%
%   \begin{itemize}
%     \item \hs{data Integer = Positive Nat | Negative Nat}
%   \end{itemize}
%
%   % so this is an interesting piece of work to hook this up
%   % with an interactive theorem prover, isa/coq/agda
%   % ... could make us spend less time on proving boring properties ;)
%
% \end{frame}

% how to jump to conclusions/rounding off here in a reasonable way?

% so I think the bottom-up approach is the right way to go,
% and why do I think this?

\begin{frame}
  \frametitle{Conclusions}
  \begin{itemize}
    \item Evaluate your programs
    % When we do a proof by hand of a functional program,
    % we rather have some insight in what a particular function does,
    % rather than looking at the definition syntactically.
    % A way to to get this intuition of what a program does for
    % a computer is of course to evaluate it!
    \pause
    \item ``Completeness'' up to a certain depth
    % and I think this is crucial to the good results we get
    \pause
    \item Progress in automated induction
  \end{itemize}
\end{frame}

\begin{frame}
\frametitle{}
\begin{center}
\hs{github.com/danr/hipspec}
\end{center}
\end{frame}

\begin{frame}[fragile]
  \frametitle{Conditionals as Functions}

  \insertlemma

\begin{verbatim}
  whenSorted :: [Nat] -> [Nat]
  whenSorted xs = if sorted xs then xs else []
\end{verbatim}

  \begin{equation*}
    \faa{\hs{x}}{\xs} \hs{sorted (insert x (whenSorted xs))} = \hs{True}
  \end{equation*}

\pause

\begin{verbatim}
  sorted (insert x (whenSorted xs))
= sorted (insert x (if sorted xs then xs else []))
= if sorted xs then sorted (insert x xs)
               else sorted (insert x [])
\end{verbatim}

\end{frame}

\begin{frame}[fragile]
  \frametitle{What is HipSpec?}
\tikzstyle{block} = [rectangle, draw=Purpleee, thick, text width=4.75em, text centered]


\makebox[\textwidth][c]{\
\begin{tikzpicture}

  \node at (0,0) [block,text width=120] (src) {\
\\
\textbf{Haskell source}
{\small
\begin{align*}
&\hs{rev [] = []} \\
&\hs{rev (x:xs)} \\
&\quad\hs{= rev xs ++ [x]} \\ \\
&\hs{prop\_rev xs} \\
&\quad\hs{= rev (rev xs) =:= xs}
\end{align*}
}
};
%\pause
  \node at (5,2) [block,text width=120] (hip) {\
\textbf{Hip}
\\
\emph{Haskell Inductive Prover}
\begin{itemize}
\item FOL translation
\item Apply induction
\item Success% \pause
, or stuck!
\end{itemize}
};
%\invisible<1-3>{
  \node at (5,-2)  [block,text width=120] (qs) {\
\textbf{QuickSpec}
\\
Eq-theory from testing:
\vspace{-0.5\baselineskip}
{\small
\begin{align*}
&\hs{rev (xs ++ ys)} \\
&\quad\hs{= rev ys ++ rev xs} \\
&\hs{xs ++ [] = []} \\
&\hs{xs ++ (ys ++ zs) =} \\
&\quad\hs{(xs ++ ys) ++ zs}
\end{align*}
}
};

 % \onslide<5>{
    \node at (8.5,-2) [block] (lemmas)
       {\textbf{HipSpec} \\ \emph{Use these as lemmas!!}};
 % }

\end{tikzpicture}
%}
}

\end{frame}

\end{document}

